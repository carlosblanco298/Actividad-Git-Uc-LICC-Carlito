\documentclass{article}
\usepackage{graphicx} % Required for inserting images
\usepackage[utf8]{inputenc}

%\title {Programa ParImpar}
%\author{Carlos Blanco}
%\date{May 2025}

\begin{document}


\begin{center}
    {\Huge \textbf{Programa ParImpar} \par}
    \rule{80mm}{0.1mm}

    {\large \textbf{Creador:}\\\normalsize Carlos Blanco \par} 
    \medskip
    \textbf{Fecha de entrega:} 12 de Mayo, 2025
    \vspace{10pt}

\end{center}

\noindent Creado en python por Carlos Blanco, usando \textit{Variables y Expresiones} y \textit{Control de Flujo (Condiciones e iteraciones)}.

\vspace{10pt}

\noindent El algoritmo para calcular si un número es par o impar, el cual fue visto en el curso de \textit{Introducción a la Programación} el primer semestre del año 2025 en la Pontificia Universidad Católica de Chile, es un clásico en el mundo de la programación, usado para dar ejemplos o crear programas más complejos.

\vspace{10pt}

 \noindent Este programa recibe como valor de entrada un número  otorgado por el usuario, luego, ejecuta la división de este valor y el número 2 para evaluar el resto. Luego de esto, decidira si el número otorgado es par o impar si el resto es o no cero, imprimiendo en consola el string "Él número \{número\_otorgado\} es par" o "Él número \{número\_otorgado\} es impar" si el número es par o impar respectivamente.

\vspace{10pt}

\section*{Funcionamiento}

\noindent En esta sección explicaremos más a detalle el funcionamiento del algoritmo.

\vspace{10pt}

\noindent Para empezar el programa, a través de un input , el cual imprimirá en consola "¿Qué número deseas saber si es par o impar?", recibira un valor del usuario.

\vspace{10pt}

\noindent Luego el programa entrará en un while en caso de que el input ingresado por el usuario no sea un número. Dentro de este while se le pedirá nuevamente al usuario que ingrese un valor numérico. Esto terminará una vez que el valor ingresado por el usuario sea un número.

\vspace{10pt}

\noindent Para finalizar, una vez terminado el while, el programa ejecutará dentro de un if la división del número entregado y el 2 es 0. En caso de que el resto de la división sea cero, el programa imprimirá en la consola, con ayuda de un print, el mensaje "El número \{número\_ingresado\} es par". En caso de que el resto de la división no sea 0, el programa entrará en un else, el cual imprimira con un print "El número \{número\_ingresado\} es impar" en el caso de que el resto de la división no sea cero.




\end{document}
